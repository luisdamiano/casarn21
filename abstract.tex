% Guidelines ---------------------------------------------------------
%
% * Due Friday, January 29, 2021
%
% * We emphasize the application of statistics in solving real-life
% problems
% 
% * data sets and analyses which motivated the work shares equal
% importance with development of the statistical theory.
% 
% * new or innovative applications of existing statistical %
% methodologies are appropriate
% 
% * modern experimental design and statistical issues in QTL mapping
% and epigenomics
% 
% * Student competition
% 
% * title length up to 130 chars with spaces.
% 
% * abstract length up to 1220 chars with spaces.
%
% Initial abstract outline -------------------------------------------
% 
% * Methodology
%
%   * What makes functional input unique?
%
%   * Description of our methodology
%   
%   * What makes our approach better?
%
%     * More constructive empowering the scientist to inject plant
%     knowledge, improving over "data driven" / "model agnostic" purely
%     statistical approaches for dimensionality reduction.
%
% * Application
%
%   * Mention Foresite
% 
%   * Mention APSIM
%
%   * Mention two or three most important functional inputs
%
%   * Why is the emulator important? what does it bring to the table?
%
% * Results & implications
%
%   * findings, results, or arguments
%
%   * significance or implications of your findings or arguments

Title: Automatic Dynamic Relevance Determination of soil properties
over different soil layers for yield prediction using APSIM

Authors: Luis Damiano, Jarad Niemi

%%%%%%%%%%%%%%%%%%%%%%%%%%%%%%%%%%%%%%%%%%%%%%%%%%%%%%%%%%%%%%%%%%%%%%
% Trimmed version 
% ~ 1220 chars

APSIM is a computer model that simulates crop growth via the
mathematical modeling of physical mechanisms. We study how the
relevance of soil properties on plant growth behaves dynamically at
different depths with a Gaussian process (GP) emulator. The GP
literature emphasizes scalar or vector inputs, which have less
potential to fully exploit the functional structure and model its
relevance; in fact, prediction and calibration would still require
soil samples for each soil layer.

We present an Automatic Dynamic Relevance Determination (ADRD) via a
parametric length-scale function of depth to capture how fast
relevance transitions over depth from a neutral state.  Prediction
can more easily adapt to natural factor variability like plant root
length. The approach naturally incorporates prior knowledge, if any,
as the transition shape and characteristics can be tuned based on
theoretical or empirical grounds. In an application, we estimate the
ADRD parameters of key soil properties in different scenarios and
explore the shape, direction, and rate of change of the input
relevance over soil layers. Inference helps to better understand the
complex biological process and inform future field data collection and
APSIM calibration efforts.

%%%%%%%%%%%%%%%%%%%%%%%%%%%%%%%%%%%%%%%%%%%%%%%%%%%%%%%%%%%%%%%%%%%%%%
% Extended version 
% ~ 2800 chars

We study how the relevance of soil properties on plant biomass growth
behaves dynamically across a range of soil layers at different depths.

The Agricultural Production Systems sIMulator (APSIM) is numerical
computer model that simulates crop growth via mathematical dynamic
models capturing the essential plant and soil physical processes
mechanisms. From exploratory analyses, we identified potentially
highly relevant biomass growth drivers such as the content of water
retained after gravitational flow, the fractional amount of water
above retention that can drain under gravity per day, and soil organic
carbon. These inputs, which are modeled over a continuous index, are
observed in field data and calibrated in the model and as a finite
collection of soil depths (arbitrary soil layer discretization).

We set up a Gaussian process (GP) emulator (AKA surrogate or
metamodel) to gain insight on how crop growth responds to the inputs
of interest. The GP literature emphasizes the modeling of collections
of scalar inputs, vector inputs, scalar-valued summaries of vector
inputs, and other projections of the inputs treated as a vector
input. These methodologies do not offer enough insight on the inputs
relevance; in fact, prediction and calibration would still require
soils to be sampled or interpolated at every soil layer no matter how
relevant it is to determine the output. Morris (2012) suggests that
the principal component representation of indexed inputs has less
potential to identify the underlying functional structure.

In a GP, the correlation between output pair is a function of the
similitude between their corresponding inputs. Extending Morris
(2012), we use a parametric length-scale function of the input index
with unknown parameter values to implement an Automatic Dynamic
Relevance Determination (ADRD). Learning the parameter values from
APSIM simulations captures how slowly or fast soil layer relevance
transitions from or towards a neutral state. Prediction can now adapt
to different input relevance over soil depth and better account for
variability from, say, a variety of plant root length. This is Indexes
at which the functional length-scale takes larger values become
irrelevant, effectively guiding future data collection and model
calibration. Additionally, our approach enables the incorporation of
prior knowledge about the underlying process as the data analyst can
its the shape and characteristic to accommodate for both previous
empirical findings or theoretical grounds.

In an application, we estimate key soil properties ADRD parameters. We
explore the shape, the direction, and rate of changes of the input
relevance to predict crop yield across soil layers. This findigs
enable a better understanding of the underlying growth process and can
be used to inform future field data collection and APSIM calibration.

%%%%%%%%%%%%%%%%%%%%%%%%%%%%%%%%%%%%%%%%%%%%%%%%%%%%%%%%%%%%%%%%%%%%%%
% Pointers and mumbling

Development of an appropriate approximating relationship between
process variables and the process response, and optimization in search
of process variable levels to produce a desirable response (eg
maximize yield).

We propose (uhm?) a functional length-scale

A surrogate for a mechanistic model.

Gathering data is expensive.

Cheaper way to explore relationships.

Complex models require more resources to calibrate and simulate.

Understanding sensitivity to inputs.

Automatic optimization.

Development of an appropriate approximating relationship between
process variables and the process response, and optimization in search
of process variable levels to produce a desirable response (eg
maximize yield).

Natural variables (expressed in their natural units of measurements)
vs. coded variables (dimensionless input transformations such as unit
cube, or zero mean variance 1).

Avoid expernsive field data collection.

* Preliminary model analyses suggest that [functional inputs 1, 2, 3]
are among the soil properties to predict yield. Their relationship
with yield is highly non-linear. These inputs are indexed by depth.

These inputs have a natural structure well studied in agronomy.

Typically, multivariate inputs are processed with a dimensionality
reduction technique such as PCA.

Considering it just a multivariate inputs losses the information
contained in the index structure.

Dimensionality reduction techniques,
even those designed for functionals like FPCA, fall short in that new
field measurements would require measuring the input variable across
all depths.

A Gaussian process regression, also known as Kriging, 

The correlation between inputs and outputs are typically captured by
the length-scale parameter (one per input variable). To account for
the inherent input structure, we replace a finite collection of
independent length-scale with a positive-valued length-scale
functional that drives the correlation between yield and the inputs
along the depth axis.

Length-scale

To better understand the relationship between inputs and outputs in a
complex model.

Benefits:

* To add agronomical prior knowledge about the underlaying process,
instead of fully data driven techniques.

* Identify depth levels that are more relevant (higher weight) to
focus future calibration and field experiments and guide future data
collection.

* More constructive approach vs.

* Better understanding of the underlying model relationship -- at
which depth are variables more impactful?

%%% Local Variables:
%%% mode: latex
%%% TeX-master: t
%%% End:
